\chapter*{Conclusion}\label{chap:conclusion}
\addcontentsline{toc}{chapter}{Conclusion}

The thesis describes the topic of the data binding used in the UI to tie the model and the view of the user interface.
The work was oriented to the Dart programming language. 

We explored existing solutions, compared and categorized them and described, why no of them is satisfactory for our conditions. 
	Therefore we created the new UI library(\tiles) in the Dart programming language, 
which is component driven (explained in \fullref{sec:existing-component}), 
independent and fully covered by unittests. 

The tiles library was created with an inspiration from the \facebook \react library wrote in the JavaScript.
It provides a tool for effective creation of the UI part of the application. 
The library was created with respect to several aspects:
\begin{description}
	\item[Pure Dart library] was fulfilled by usage of no template engine 
		or other tool describing the UI in the different type of the information.
	\item[Testability] was the major aspect of the implementation. 
		We perform several architectural decisions toward the testability of the code using our library like
		the \textbf{separation} of the rendering logic from the main class \texttt{Component} used by the user of the library
		and no template usage.
	\item[Easy to use API] was reached by the separation of the \texttt{Component} class 
		and by creation of methods for frequently performed tasks.
\end{description}

As we wrote in the \fullref{chap:oursolution}, the first step of the work on the thesis was 
to create the port of the \react library into the Dart language. 
This port was "slow" in some cases, therefore we created mentioned \tiles library.
The usefulness of it was shown in the \fullref{chap:benchmarks}, which shows that our effort was not wasteful.

One of the most important results of the thesis is the UI \tiles library, which was described in the text of the thesis.
