\chapter{Existing solutions}\label{chap:existing}

There are two main approaches in data-binding based on main concepts of rendering content, template driven and component driven.
These two concepts implements data-binding by different architecture design. 

\section{Template driven}\label{sec:existing-template}

	Template driven approach is, as name predicts, based on usage of some template engine. 
	These template engines take template and data and create HTML representation of passed data in the form of the template. 
	They can be considered as a function $t:\mathbb D\mapsto\mathbb H$ where $\mathbb D$ is a set of all possible data and $\mathbb H$ is a context-free language of valid HTML.

	An easy example of a template, for example using \textit{handlebars.js} can look like this one (from \textit{handlebars.js} website):
  \input{content/examples/handlebars_entry.tex}

	When programmer want to use this template, he should create data object, which minimal version in JSON format is in next example:
  \begin{minted}{json}
{
  "title": "Some title",
  "body": "This is the content of the page"
}
\end{minted}


	When template is filled by this data, following HTML will be produced
  \begin{minted}{html}
<div class="entry">
  <h1>Some title</h1>
  <div class="body">
    This is the content of the page
  </div>
</div>
\end{minted}


	Most of template engines also offer logic markup:
  \begin{minted}{html+django}
<h1>Comments</h1>

<div id="comments">
  {{#each comments}}
  <div class="entry">
    {{#if author}}
      <h1>{{firstName}} {{lastName}}</h1>
    {{else}}
      <h1>Unknown Author</h1>
    {{/if}}
    <div>{{body}}</div>
  </div>
  {{/each}}
</div>

\end{minted}


	Template driven view are highly used because of they looks like resulting HTML. 
	It is easy to convert HTML produced by graphic designer to template used in source code. 
	Also programmers used to work in HTML more easily write templates then some other representation of the view.

	Different libraries work with templates in different way. 
	Some of them really parse input template as a string, recognize DOM tree in it and work with them that way.
	Others uses in-browser html parser to parse template and then fill it with data.
	This approach, because of this usage of tools accessible from browser, is more difficult to render on the server.

	Table~\nameref{table:template-driven-libraries} compares some of existing solutions which are standalone libraries or MVC frameworks.

	\begin{table}
		\begin{tabular}{|l|l|c|c|c|c|}
			\hline
			\textbf{Solution}& \textbf{Language}   & \textbf{Standalone} & \textbf{In browser} & \textbf{On server} & \textbf{Tricky on server}\\
			\hline
			handlebars       & JavaScript          &         yes         &        yes          &        yes         &           no             \\
			\hline
			{{mustache}}     & JavaScript, python...&        yes         &        yes          &        yes         &           no             \\
			\hline
			dust             & JavaScript          &         yes         &        yes          &        yes         &           no             \\
			\hline
			AngularJS        & JavaScript          &         no          &        yes          &        no          &           yes            \\
			\hline
			meteor           & JavaScript          &         no          &        yes          &        no          &           yes            \\
			\hline
			EmberJS          & JavaScript          &         no          &        yes          &        yes         &           no             \\
			\hline
			Derby            & JavaScript          &         no          &        yes          &        yes         &           no             \\
			\hline
			Polymer          & JavaScript          &         yes         &        yes          &        no          &           yes            \\
			\hline
			Polymer.dart     & Dart                &         yes         &        yes          &      not now       &           yes            \\
			\hline
		\end{tabular}
		\caption{Comparison of template driven libraries}
		\label{table:template-driven-libraries}
	\end{table}

\section{Component driven}\label{sec:existing-component}

	Component driven views, in opposite to template driven, don't use any additional type of data like templates. 
	Whole view representation of data is stored in structure of so called \textit{components}, 
	which are native objects connected to tree structure to represent DOM. 

	When tree of components is constructed, it is rendered to DOM by depth-first search of the component tree.
	When components and HTML elements are connected by stored association, 
	every change in component structure can be applied to DOM tree. 

	In addition, if we have tree of components, we can easily, by similar depth-first search, 
	create markup string representing HTML of the component tree.
	By this we can render whole component tree on server.

	For this purpose programmer should not use browser specific features in part of component, 
	which is called in initial built of component tree.

	\begin{table}
		\begin{tabular}{|l|l|c|c|c|c|}
			\hline
			\textbf{Solution}& \textbf{Language}   & \textbf{Standalone} & \textbf{In browser} & \textbf{On server} & \textbf{Tricky on server}\\
			\hline
			React            & JavaScript          &         yes         &        yes          &        yes         &           no             \\
			\hline
		\end{tabular}
		\caption{Comparison of component driven libraries}
		\label{table:component-driven-libraries}
	\end{table}

	We decided to use similar approach as is used in React library, so we briefly describe it.

	\subsection{React}\label{subsec:existing-component-react}
		
		React is JavaScript UI library from facebook. 
		Its main concept is to pack parts of the web application into reusable components, 
		which are represented as object in JavaScript. 

		This components can be mounted into elements in DOM, for now, we will call it \textit{mount root}. 
		This will create \textit{virtual DOM} "mounted" to \textit{mount root}. 
		This virtual DOM is then reflexed into the real DOM under the \textit{mount root}.

		Components are organized to virtual DOM tree, where data flows from root component to leaves. 
		This data flow is implemented by props of the component, which are read-only.
		Component have also own state, which should be stored in state attribute and updated by methods \texttt{setState} and \texttt{replaceState}. 
		State shouldn't be updated directly to preserve invariant, that real DOM always represents actual state of the virtual one. 

		Component create structure under it by its method \texttt{render}, which should return one instance of a component, 
		which will be the child of this component.
		In render function, it can add to this rendered component props, which is the way, how data is flowing down, 
		and children, which is the way, how to create spreading tree, not just a line. 
		If the child component get children, it will be in \texttt{children} attribute. Component can reuse them in \texttt{render} method or ignore them.

		This is because React maps all components to HTML elements and want to create more than one tree children only in components representing real DOM elements.

		Components can listen to events on the DOM components (internal react components, representing DOM elements). 
		They are attached trough props by event listeners. 

		When event occurs, it bubble up to \textit{mount root} where is caught by React.
		Then React simulate event bubbling from DOM component, which represents target element in virtual DOM,
		with synthetic event, which manage browser compatibility. 

		When this event is caught by some custom component, this component can react on that. 
		It can change state, call some functions, store some data or what ever it wants.

		State change(by mentioned methods) trigger redrawing of virtual DOM. 
		This will use render methods to create new children of the components 
		and process whole tree by depth-first search, 
		which produce list of changes needed to get real DOM to state representing virtual one.

		For this purpose, React implements some component life-cycle methods, which we will not discuss here. 
		They are the superset of life-cycle methods we implement in Tiles library. 

		For more information about react and it's architecture and API 
		reader can go to the website of the React project \url[http://facebook.github.io/react/]

\section{Conclusion}\label{sec:existing-conclusion}
		
		Here should be described, why we decided to use component driven approach and take inspiration from React library.
