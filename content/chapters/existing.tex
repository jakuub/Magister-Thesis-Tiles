\chapter{Existing solutions}\label{chap:existing}

When we think about the building of the user interfaces, we can think about the building them from components.
The component is a part of the UI, which has a functionality, own look and maybe some interaction.

The HTML is a basic component structure. Every element is a component, all elements are composed into tree structure. 
Elements have some functionality, own look(e.g. image) and some of them have interactions(e.g. input).

So when a library want to bind the data with a view, it basically bind the data to a component.

There are different approaches of the creating these components and the connection between them and a data. 
Components can be created directly by a programming language (Component driven), 
or by using a template engine, which create components based on the template, 
which describe component structure, in the most cases by HTML-like syntax(Template driven).

These two approaches do the same thing, create structure of the components, different way.

\section{Template driven}\label{sec:existing-template}

	Template driven approach is, as the name predicts, based on the usage of a template engine. 
	Template engines take a template and the data and create a component structure, 
	which is reflected into the HTML representation of passed data in the form of the template. 
	They can be considered as a function $t:\mathbb D\mapsto\mathbb H$, 
	where $\mathbb D$ is a set of all possible data and $\mathbb H$ is a context-free language of valid HTML.

	An easy example of a template, for example using \textit{handlebars.js} can look like this one (from \textit{handlebars.js} website):
  \input{content/examples/handlebars_entry.tex}

	When programmer want to use this template, he should create data object, which minimal version in JSON format is in next example:
  \begin{minted}{json}
{
  "title": "Some title",
  "body": "This is the content of the page"
}
\end{minted}


	When template is filled by this data, following HTML will be produced
  \begin{minted}{html}
<div class="entry">
  <h1>Some title</h1>
  <div class="body">
    This is the content of the page
  </div>
</div>
\end{minted}


	Most of template engines also offer logic markup, which add possibility of the better control of a composed structure. 
	This is highly usable when programmer want to create more complex structures based on the data. 
	The typical example of this structure is the \texttt{<ul>} list generated from the array of items to render.
	
	This "in template" logic has on one hand some advantages, on the other hand, 
	the HTML syntax was not created to represent a logic, but an information.
	Because of this, more complex templates witch not so trivial logic in it becomes hard to read and understand.
	
	Easy use of the logic in the template is shown on the next example:
  \begin{minted}{html+django}
<h1>Comments</h1>

<div id="comments">
  {{#each comments}}
  <div class="entry">
    {{#if author}}
      <h1>{{firstName}} {{lastName}}</h1>
    {{else}}
      <h1>Unknown Author</h1>
    {{/if}}
    <div>{{body}}</div>
  </div>
  {{/each}}
</div>

\end{minted}


	The template driven view are highly used because of the syntax similarity between the template and the resulting HTML. 
	It is easy to convert the HTML produced by a graphic designer to the template used in the source code. 
	Also programmers used to work in the HTML more easily write templates then some other representation of the component structure.

	Different libraries work with templates in a different way. 
	Some of them really parse the input template as a string, recognize component tree in it and work with the template that way.
	Others uses in-browser HTML parser to parse the template and then fill it with the data.
	This approach, because of its usage of tools accessible only from the browser, is more difficult to render on the server.

	Templates are mostly used two different ways:
	\begin{description}
		\item[Template used as View] \hfill \\
			Template is used to render HTML structure into some element. 
			Functionality of the HTML structure is then realized separately and attached to it.  
			This is used for example in the CanJS.
		\item[Template used as Component] \hfill \\
			The other (and more modern) use of template is to represent one component with attached functionality, 
			which can be represented later as custom HTML element in other templates. 
			In this template, other custom components can be created by using their custom HTML element representation. 
			It is not necessary to create them separately in the code of an application.

			This approach is used e.g. in Polymer project, which work with so called "shadow DOM" which use similar concept.
	\end{description}

	\fullref{table:template-driven-libraries} compares some of existing solutions which are standalone libraries or MVC frameworks.
	The aspects of the comparison are a natural rendering in the browser and  on the server and if the library is a standalone UI library, 
	or is a part of the more complex MVC framework. 
	We don't compare a possibility to render the view on the server other then the natural way, 
	because it is always possible to render it by usage of tools like the \textit{PhantomJS}.

	\begin{table}
		\begin{tabular}{|l|l|c|c|c|c|}
			\hline
			\textbf{Solution}& \textbf{Language}   & \textbf{Standalone} & \textbf{In browser} & \textbf{On server} \\
			\hline
			handlebars       & JavaScript          &         yes         &        yes          &        yes         \\
			\hline
			{{mustache}}     & \shortstack{JavaScript,  
														\\		python...} &        yes         &        yes          &        yes          \\
			\hline
			dust             & JavaScript          &         yes         &        yes          &        yes         \\
			\hline
			AngularJS        & JavaScript          &         no          &        yes          &        no          \\
			\hline
			meteor           & JavaScript          &         no          &        yes          &        no          \\
			\hline
			EmberJS          & JavaScript          &         no          &        yes          &        yes         \\
			\hline
			Derby            & JavaScript          &         no          &        yes          &        yes         \\
			\hline
			Polymer          & JavaScript          &         yes         &        yes          &        no          \\
			\hline
			Polymer.dart     & Dart                &         yes         &        yes          &      not now       \\
			\hline
		\end{tabular}
		\caption{Comparison of template driven libraries}
		\label{table:template-driven-libraries}
	\end{table}

\section{Component driven}\label{sec:existing-component}

	Component driven views, in opposite to the template driven, don't use any additional type of data like templates. 
	Components are created by the same programming language as the functionality and 
	are composed into the tree structure which is mapped into the DOM. 

	When the tree of components (we will call it "Virtual DOM" later)is constructed, 
	it is rendered to the DOM by the depth-first search of the component tree.
	When components and HTML elements are connected by stored associations, 
	every change in the component structure can be applied to the DOM tree. 

	In addition, if we have the tree of components, we can easily, by the similar depth-first search, 
	create the markup string representing the HTML markup of the component tree.
	This enable the rendering of the whole component tree on the server without use of browser-specific features.

	An example of the component driven UI library is the JavaScript library \textit{React} created by the facebook.
  \textit{React} is standalone UI library which enable native rendering of the component structure as in the browser, so on the server.

	We decided to use a similar approach to \textit{React} library, so we briefly describe it.

	\subsection{React}\label{subsec:existing-component-react}
		
		\textit{React} is JavaScript UI library from facebook. 
		Its main concept is to pack parts of the web application into reusable components, 
		which are represented as object in JavaScript. 

		This components can be mounted into elements in DOM, for now, we will call it \textit{mount root}. 
		This will create \textit{virtual DOM} "mounted" to \textit{mount root}. 
		This virtual DOM is then reflexed into the real DOM under the \textit{mount root}.

		Components are organized to virtual DOM tree, where data flows from root component to leaves. 
		This data flow is implemented by props of the component, which are read-only.
		Component have also own state, which should be stored in state attribute and updated by methods \texttt{setState} and \texttt{replaceState}. 
		State shouldn't be updated directly to preserve invariant, that real DOM always represents actual state of the virtual one. 

		Component create structure under it by its method \texttt{render}, which should return one instance of a component, 
		which will be the child of this component.
		In render function, it can add to this rendered component props, which is the way, how data is flowing down, 
		and children, which is the way, how to create spreading tree, not just a line. 
		If the child component get children, it will be in \texttt{children} attribute. Component can reuse them in \texttt{render} method or ignore them.

		This is because \textit{React} maps all components to HTML elements and want to create more than one tree children only in components representing real DOM elements.

		Components can listen to events on the DOM components (internal react components, representing DOM elements). 
		They are attached trough props by event listeners. 

		When event occurs, it bubble up to \textit{mount root} where is caught by \textit{React}.
		Then \textit{React} simulate event bubbling from DOM component, which represents target element in virtual DOM,
		with synthetic event, which manage browser compatibility. 

		When this event is caught by some custom component, this component can react on that. 
		It can change state, call some functions, store some data or what ever it wants.

		State change(by mentioned methods) trigger redrawing of virtual DOM. 
		This will use render methods to create new children of the components 
		and process whole tree by depth-first search, 
		which produce list of changes needed to get real DOM to state representing virtual one.

		For this purpose, \textit{React} implements some component life-cycle methods, which we will not discuss here. 
		They are the superset of life-cycle methods we implement in Tiles library. 

		For more information about react and it's architecture and API 
		reader can go to the website of the \textit{React} project \url[http://facebook.github.io/react/]

\section{Conclusion}\label{sec:existing-conclusion}
		
\todo{Here should be described, why we decided to use component driven approach and take inspiration from \textit{React} library.}
