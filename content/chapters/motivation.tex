\chapter{Motivation}\label{chap:motivation}

The first question, as always should be, is the motivation of this work. 
What is the motivation to create another library, that will handle data binding in Dart?

The motivation to create the \textbf{tiles} library contains from several aspects, 
which are not contain in no other \texttt{Dart} library
\footnote{We didn't find any suitable library and didn't hear about it}.

\begin{description}
	\item[Dart as a programming language] \hfill \\
		Dart language is young programming language with an active development and progress. 
		One of its advantages is optional typing, 
		the build-in compilation to the JavaScript, which enable programming a browser applications,
		and a Java-like virtual machine, which runs the Dart in the most commonly used operating systems.

		It is designed for the web applications with all necessary support for them. 
		As it enable the compilation into the JavaScript and running directly under the OS, 
		it also enable to share a source code between server part of an application 
		and its client application running in the web browser.

		Dart also guarantees browser compatibility, what is important for ease of web application development.
	\item[Testability] \hfill \\
		Very important aspect in a building complex application is the testability of the source code.

		Because of this, it is essential to use libraries, which enables easy testing and mocking components.
	\item[Server side rendering] \hfill \\
		Server side rendering is very important for user experience and for search engine optimization. 

		When we have the CPM\footnote{Computer Programming Language} which can be used as on the server, so in the clients browser, 
		it is natural to think about a use of the same source code to create an in-browser application, 
		and to render its page on the server. 
	\item[No templates] \hfill \\
		This aspect is important from two point of views: the testability and the server side rendering.

		From the testability point of view, it is easier to test and mock structures created in only one CPM. 
		If the template is used to create a component of the UI, 
		it is much more difficult to test it and also think about this testing.
		If this component is only one class in the CPM without dependences on another type of the information, 
		testing is more natural and easier to think about.

		From the server side rendering point of view, if we want to work with templates, 
		we need to access them differently when we work in the OS and in the clients browser.
		If we have the structure fully composed in one CPM, it is easier to compose the same HTML structure on the server as in the browser, 
		then if we have the structure composed by the template and the CPM.
	\item[Only one language] \hfill \\
		This aspect is very related with the previous one. 
		When the application is created fully in one programming language, 
		it is easier for programmers to work with it (they don't have to switch between different CPM).

		Also it is easier to compile whole application into the JavaScript, analyze the source code or refactor it.
	\item[Reliability] \hfill \\
		The reliability has significant importance in complex applications. 
		This reliability can be achieved by automatic tests, a robust design and quality development. 
\end{description}

When we take into account these aspects, there exists no library, which fulfill all of them.

As there is a need for this kind of a library, we decided to design and create one. 
