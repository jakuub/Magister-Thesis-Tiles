\chapter{Performance}\label{chap:performance}

  The performance is an important aspect of the UI library\footnote{as almost everywhere}.
  If the UI library is slow, the user experience is unsatisfactory and application is considered as worse as it can be.

  The question about performance of the UI library can be divided into the initial render, 
  the update of the UI and the continuous consumption of computational resources.

  The complexity of each task of the UI can be dependent on data size and resulting structure size. 
  These two aspect are generally connected. 

  We will take into account both of them. 
  Data are a partial input of all tasks. 
  The resulting structure combine the input data and the complexity of the UI. 
  Therefore "How long it takes to process the structure of a size N" is in some cases a good question.

  The complexity of the application logic not reflected into the complexity and a size of the UI structure is not relevant for as, 
  because it is independent from the UI library. Therefore we will suppose, that it is constant. 

  \section{Initial render}\label{sec:performance-initial-render}
    
    The initial render creates the whole structure of the UI, 
    therefor it is not possible to perform it in less then a linear time from the size of the resulting structure.
    It should be at least constructed.

    The performance is also at least linear from the size of the read data(they must be at least read).

    The \tiles library constructs virtual DOM representing the UI with a tree of the nodes. 
    Each node contain the constant size of the information used by the \tiles library. 

    The creation of the virtual DOM will process each node in the virtual DOM tree constant number of times. 
    We assume a constant complexity of the application logic, so every life-cycle method called by our library will take a constant time.
    Therefore the creation of the virtual DOM is linear from the size of it.

    The initial render will process the virtual DOM by the depth-first search 
    and creates the DOM structure in the process of the search without any looping and repeating.
    So the projection from the virtual DOM into the real one is performed in linear time from the size of the virtual DOM.

    At the summary, the whole initial mount consist of the creation of the virtual DOM (linear from the size of the virtual DOM), 
    and the projection of the virtual DOM into the real one (also linear from the size of the virtual DOM).
    The complexity of the initial render is therefore linear from the size of the virtual DOM.

    The last question is, what relation is between the size of the virtual and the real DOM. 

    The subtree without any DOM component in it is useless from the UI point of view, 
    so we assume, that every leaf\footnote{A leaf is also a subtree} of the virtual DOM is a DOM component, 
    which will be linearly reflected to the HTML element.

    A long line of not DOM nodes without any branching is also useless (it can be replaced by one virtual DOM).

    From these facts is obvious, that in the most cases, the ratio of DOM components in the virtual DOM has the constant lower bound.

    So the complexity of the initial render is linear from the size of the virtual DOM, 
    which is generally linear from the size of the resulting structure.

  \section{Update of the UI}\label{sec:performance-update}

    The performance of the update process is reflected into the agility of the UI of the application.
    We assume, that the library is used wisely by a user of it.
    We also suppose no usage and user optimization by \texttt{shouldUpdate} method.

    In the \tiles library, when data are changed and the redraw method was called, 
    the competent node is marked as dirty. 
    The update process will redraw the whole virtual DOM structure under the dirty node. 
    The list of changes is produced and subsequently applied to the real DOM. 

    The complexity of the update process is therefore asymptoticly equal 
    to the size of the subtree of the virtual DOM with the root in the dirty node.
    This is in the worse case the whole virtual DOM. 

    This complexity can be decreased by proper usage of the \texttt{shouldUpdate} function. 

    In general scenario, parts of the UI dependent on a data change are not huge, 
    which decreases the time needed for the update.  

  \section{Continuous resource consumption}\label{sec:performance-continuous}
    
    As we perform an asynchronous updating of the virtual DOM, the only one repeated task is to check, 
    if there is something to be updated. This is not a performance bug, it will be discussed in the \fullref{sec:performance-optimizations}.

    The check of the update need is performed on each mounted virtual DOM in the constant time(root node isDirty or hasDirtyDescendant). 
    When we have the whole application UI represented in one virtual DOM, there is only constant time to check the update need. 

    In general, the complexity of the task, which is performed periodically in the time, is linear from the number of virtual DOMs. 
    Virtual DOMs should not be initialized dynamically from the size of a data, for this purpose should be used the virtual DOM itself.
    Therefore we assume the constant number of virtual DOMs in the application, 
    so the constant complexity of the task performed in each animation frame.

  \section{Performance optimizations}\label{sec:performance-optimizations}
    
    The performance of the pure algorithm can be enhanced by an optimization of selected parts. 
    For this purpose we used batched updates, the up-down update (from the root tree to the leafs of the virtual DOM) 
    and many small refinements like \texttt{shouldUpdate} method. 

    \subsection{Batched updates}\label{sec:performance-optimizations-batched-updates}
   
      Every triggered redraw of the component will cause an event in the \texttt{needUpdate} stream. 
      The event is caught by the node containing the component, which mark itself as dirty. The synchronized process of redraw ends here. 

      If more than one redraw happened, all of them are batched by marking a components node as dirty.
      Mark the node as dirty cause adding the flag to the route from the node to the root node of the virtual DOM as \textit{have dirty descendants}.

      On every animation frame, each root node is check for the need of the update, and if it is dirty or has dirty descendant, the update is started.
      The update perform all needed updates in the virtual DOM in one batch.
      The update is perform only when the real DOM will be drawn to the user
      \footnote{The information about redrawing the UI by the browser is implemented by the \textit{animation frame} event}. 

    \subsection{Up-down update process}\label{sec:performance-optimizations-batched-updates}

      We can perform the update process of bathed updates from the root to leafs of the virtual DOM tree. 
      Therefore when the node is deleted from the virtual DOM, none of it's descendants will be updated, even if they wanted to.
      The update of removed node would be useless, because it's change will be never reflected into the real DOM. 

      The up-down update also saves the performance from repeating the update of the same node. 
      For example, we have 4 nodes A, B, C and D, when the A is parent of the B, the B of the C, etc. 
      If they trigger the redraw from the bottom to the root, the order is D, C, B and A. 
      If we perform updates in the order in which the redraw was triggered, we will perform following updates:

      \begin{enumerate}
            \item update D,
            \item update C, and also D, as it got a new props,
            \item update B, so update the C and D too and 
            \item update A, which cause the update of the B, the C and the D too. 
      \end{enumerate}

      When we perform update process of the batched updates from the root to leafs, every node will be updated only once. 

      \begin{enumerate}
            \item update A,
            \item update B,  
            \item update C and 
            \item update D.
      \end{enumerate}

