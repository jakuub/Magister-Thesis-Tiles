\chapter{Data binding}\label{chap:databinding}

	Data binding is the process of tying the data in one object to another object. 
	It provides a convenient way to pass data between the different layers of the application.\cite{flexdatabinding}
	When we talk about the data binding in the UI of the web application: 
	Data binding is the process that establishes a connection between the application UI (User Interface) and business logic.\cite{databindingwikipedia}
	The application logic of the web application is represented by a data model(JavaScript/Dart used on the website) 
	and the UI by the HTML view of the data.

	The data binding is widely used in frameworks using MVC patterns. 
	The most common way to implement the data binding is by using a template engine. 
	The template engine takes the template, merge it with data and produce the view of the UI. 

	The model(data) can be connected with the view in a one direction, $model \mapsto view$, 
	or in both directions, $model \mapsto view$ and $view \mapsto model$. 
	The single way $view \mapsto model$ is not used. 

	\section{One way data binding}\label{sec:onewaydatabinding}

		The one way data binding implements the $model \mapsto view$ direction. 
		
		The server side rendering is an example, 
		where the one direction data binding is the perfect way to map data into the view.
		From the server side rendering point of view, 
		the other direction of the data binding doesn't event exist.

		The more interesting usage of the one way binding direction is in the browser application. 
		The one way data binding is used to reflect the model of the application into the UI view.
		This reflection is maintained continuously and keep the UI in the sync with the model. 
		Therefore each change of the data model is reflected into the view. 

	\section{Two way data binding}

		Two way data binding extends the one way by implementing the second direction: $view \mapsto model$. 

		One direction, when the model is updated, the view is modified to reflect the data.
		In the second direction, when the view is changed (e.g. value in the input field), 
		the model is update to represent actual view. 
		This way, the model and the view are continually synchronized.
		The model is the \textit{single-source-of-truth} for the application state\cite{angular}.

		The most common understanding of the two way data binding is the reflection of the change of input elements
		(input, select, text-area etc.) into the data model.

		There are additional informations stored in the view, which can be reflected into the model.
		For example the order of items in a list, a position in the DOM or any other information stored by the structure of the view, 
		which can be changed e.g. by drag-and-drop actions. 
		In the majority of two way data binding implementations, this type of the information isn't reflected.

		As the information from input elements can be reflected to the model by event listeners, 
		we decided to not implement the two way data binding in the first phase of the \tiles library.
		We plan to implement it later, by using the observable pattern.
