\chapter*{Introduction}\label{chap:intro}
\addcontentsline{toc}{chapter}{Introduction}

In the November 2014 on the Devoxx conference in Belgium, 
Google released the \textbf{Dart language} in form of Google Dart SDK 1.0.
The Dart language has an intention to replace JavaScript as the common language of web development on the open web platform.
\cite{dartlaunch}

The Dart programming language has possibility of compilation into the JavaScript. 
It also has own virtual machine which can be standalone or embed into the browser.
It has advantages like optional typing, 
well defined dependency management by \textit{pub} command which also enable git dependences and much more.

In the 29th of May 2013 \facebook developers released the \textbf{\react} library as an open-source. 
The \react library offer different perspective of the UI development in JavaScript applications. 
It don't use templates, enable declarative thinking in the programming of the UI.
We will provide deeper overview of the \react library in the \fullref{subsec:existing-component-react} of the \fullref{chap:existing}.

Two independent interesting solutions with lot of advantages incurred. 
The natural question is if there exist a solution which integrate the \react with the Dart language and combine their advantages.
We discovered, that there existed no integration yet, therefore we decided to create one. 

The first step was to create the port of the React into the Dart language.
It turned out, that the port is slow for a great number of custom components. 
The reason was the communication in the form of life cycle methods 
between the \react implemented in the JavaScript and the application implemented in Dart language.

That was the reason to create own Dart library with the core idea inspired by the \react library. 
The Dart library was designed, implemented and tested. 
The name is \tiles and the source code is open-source hosted on \url{https://github.com/cleandart/tiles}.

In the scope of this work we described the data binding theory, 
examined existing solutions implementing data binding, 
designed and described the \tiles library and the performance of it.

